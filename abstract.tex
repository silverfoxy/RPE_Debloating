As software becomes increasingly complex, its attack surface expands enabling
the exploitation of a wide range of vulnerabilities. Web applications are no
exception since modern HTML5 standards and the ever-increasing capabilities
of JavaScript are utilized to build rich web applications, often subsuming
the need for traditional desktop applications. One possible way of handling
this increased complexity is through the process of software debloating, i.e.,
the removal not only of dead code but also of code corresponding to features
that a specific set of users do not require. Even though debloating has been
successfully applied on operating systems, libraries, and compiled programs,
its applicability on web applications has not yet been investigated.

In this report, we present the first analysis of the security benefits of
debloating web applications. We focus on four popular PHP applications and
we dynamically exercise them to obtain information about the server-side code
that executes as a result of client-side requests. We evaluate two different
debloating strategies (file-level debloating and function-level debloating)
and we show that we can produce functional web applications that are 46\%
smaller than their original versions and exhibit half their original cyclomatic
complexity. Moreover, our results show that the process of debloating removes
code associated with tens of historical vulnerabilities and further shrinks
a web application's attack surface by removing unnecessary external packages
and abusable PHP gadgets.

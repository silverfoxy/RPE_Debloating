\section{Conclusion}
In this project, we analyzed the impact of removing unnecessary code in modern
web applications through a process called \textit{software debloating}.
We presented the pipeline details of the end-to-end, modular debloating
framework that we designed and implemented, allowing us to record how a
PHP application is used and what server-side code is triggered as a result of
client-side requests. After retrieving code-coverage information, our debloating
framework removes unused parts of
an application using file-level and function-level debloating.

By evaluating
our framework on four popular PHP applications (phpMyAdmin, MediaWiki,
Magento, and WordPress) we witnessed the clear security benefits of debloating web
applications. We observed a significant LLOC decrease ranging between
9\% to 64\% for file-level debloating and up to an additional 24\% with
function-level debloating. Next, we showed that external packages are one
of the primary source of bloat as our debloating framework was able to remove
more than 84\% of unused code in versions that used Composer, PHP's most popular
package manager. By quantifying the removal of code associated with critical
CVEs, we observed a reduction of up to 60\% of high-impact, historical vulnerabilities.
Finally, we showed that the process of debloating also removes
instructions and classes that are the primary sources for attackers to build
gadgets and perform POI attacks.

Our results demonstrate that debloating web applications
provides tangible security benefits and therefore should be seriously
considered as a practical way of reducing the attack surface of
web-applications deployments.
